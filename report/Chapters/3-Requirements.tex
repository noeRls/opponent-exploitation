\chapter{Requirements} \label{Requirement}

\lhead{Chapter 3. \emph{Requirements}}

\section{Project Goals}
The goal of this project is to demonstrate that opponent modeling is possible with neural network and to assess the performances. The performance and scalability of the algorithm will highly depend on the architecture used. Therefore, we are going to iterate multiple times on different architecture to find one that is scalable and have good performance.

We are aiming to have an algorithm that is not game-dependent and which is working with information that is available or emulatable for every imperfect game. This information is for example equilibrium and the game result. This will allow this algorithm to be re-used for other use cases.

We will at first works on a small game state to makes our experimentation. It would allow us to iterate faster and have rapid feedback on the architecture performance. We will use Kuhn poker which is an extremely simple version of poker \citep{kuhn1950simplified}.

Then we will try to scale on a larger game state, the aimed game state size is the one of poker limit texas holdem \( 10^{14}\).

Given the result and time left, we might want to increase the scope and study the performance with game state abstraction. It is particularly interesting to study as game abstraction directly impacts the scalability of an algorithm. We believe that our algorithm is likely to works better with game abstraction because the game space will be smaller and so easier for the neural network to train.

\section{Stakeholders}
Stakeholders are listed in table \ref{table:stakeholders}.
\begin{table}[ht]
    \begin{tabular}{cc}
         Stakeholder & Outcome expectation \\
         \hline
         Project Author & Successful project \\
         Project Supervisor & Successful project \\
         Research community & Serious work that can be re-used and open reflection \\
         HWU & Presentable project, university reputation \\
         Industry & Add or Improve opponent modelling in imperfect game use-case
    \end{tabular}
    \caption{Stakeholders}
    \label{table:stakeholders}
\end{table}

\section{Priorities}
This project contains a lot of works and the core part is hard to plan, more details in section \ref{subsection:risk-assesment}. This is why we prefer to have a reduced scope and increase it over time depending on the progression. To keep track of the scope we define our priorities and requirement using the MoSCoW technique. It defines multiple levels of requirements: Must (minimal usable subset), Should (high-priority requirement), Could (desirable requirement), and Won't (will not satisfy / future work). You can find these requirements and priorities in the table \ref{table:priorities}.

\begin{table}[ht]
    \centering
    \begin{tabular}{p{5cm}|p{5cm}|c|c}
         Title & Description & MoSCoW  \\
         \hline
        Implement a poker environment & A poker environment should generate and handle a game state with a set of functions to facilitate the algorithm development and testing.  & Must \\
        \hline
        Find / Compute equilibrium for Kuhan poker & & Must \\
        \hline
        Implement opponent modeling & Implement the neural networks to have a functional opponent modeling & Must \\
        \hline
        Test the algorithm & This will require the implementation/use of all the evaluation process described in section \ref{requirement:evaluation}. & Must \\
        \hline
        Automatise the testing of the algorithm & Automating the testing will enable to iterate way faster & Should \\
        \hline
        Test multiple implementations & Test multiple ways of implementing a recurrent neural network to model the opponent's behavior. Compare the result between them. & Should \\
        \hline
        Scale the algorithm to work on a large game state & Train the algorithm on poker Texas Hold'em Limit variant. It will likely be needed to rework part of the architecture. & Could \\
        \hline
        Test abstraction on a large game state & Implement poker-based abstractions and compare the result with a non-abstracted version. & Could \\
        \hline
        Test the algorithm with real person & & Won't
        % TODO won't
    \end{tabular}
    \caption{Requirements priorities}
    \label{table:priorities}
\end{table}

\section{Evaluation} \label{requirement:evaluation}
The evaluation is a crucial process for this research. It is the only way to demonstrate the capacity of neural network to model opponent behavior. Better the testing will be, the better these results will be generalized, without the need to run more experiments.

We will use poker to run our experiments. It is the most used game in imperfect game solving research as it includes complex behavior and strategy. The poker variant used will depend on the size of the game state we want to test. It will likely be Kuhn poker and texas limit hold'em.

Performances will be measured in milli-big blinds per hand (mbb/hand), the average number of big blinds win per 1000 hands. This is a common unit in poker research. Following this standard will allow these results to be comparable with other research papers.

\subsection{Exploitation testing}
The exploitation testing will ensure that the algorithm successfully extracts simple opponent behavior. Even "stupid" opponents can be interesting to study because they can give metrics on how fast the algorithm exploits them.

Here is the list of the different algorithms:
\begin{itemize}
    \item Always fold
    \item Always rise
    \item Random: plays random moves
\end{itemize}

\subsection{Exploitability testing}
Exploitability testing will provide information on how much the algorithm is exploitable.

It will play against an opponent following the equilibrium strategy. Given that, we will be able to see how reasonable our algorithm is and how well it managed to not fall into trying to exploit the equilibrium.

It will also be evaluated using the true best response with access to the private game state. In other words, at each step, the true best response will be computed with access to the card of every player. This is a non-realistic testing because it is cheating but it provides an interesting insight into how much the algorithm is exploitable.

We will also evaluate how far the algorithm is flexible and able to adapt if the opponent drastically switch his strategy. To do so we will have an opponent playing random for 100 games then playing the true best response for the remaining of the day.

\subsection{Advanced opponent testing}
In this testing category, we are trying to reproduce real-world poker examples. We will challenge our algorithm with different sub-optimal opponents.

% The list of algorithms is not fixed yet because it requires more research on which poker algorithms are open source and so free to use or which poker algorithm can be easily implemented.

We will use made up sub-optimal strategy by playing each action with a probability chosen uniformly randomly within 0.2 of the equilibrium probability. This type of sub-optimal opponent has been used in previous research in opponent modelling \citep{ganzfried2015safe}. It will be easier to compare the results.

We discard the usage of human testing for this research because it will require financial resources to have non-bias and conclusive results.

\section{Risk assessment} \label{subsection:risk-assesment}
This project has multiple risks when it comes to scaling up the algorithm. It will likely face issues with a large game state and will require an architecture rework or work on abstraction. The scale-up risk has been taken into account by reducing the impact on this research. If it is not possible to scale up the algorithm on time, we will still be able to do make conclusions about opponent exploitation with NN.

You can find the list of risks in the table \ref{tab:risk-assesment}.

\begin{table}[ht]
    \centering
    \begin{tabular}{p{5cm}|c|c|p{5cm}}
        Risk & Likelihod & Impact & Mitigation \\
        \hline
        Trouble to scale up the algorithm for poker Texas Limit Hold'em & High & Medium & Find another imperfect game with a smaller game state \\ 
        \hline
        Trouble to train the neural network & Medium & High & Change the learning method, for example from Evolutionary Algorithm to Reinforcement Learning. If not sufficient make research about it and reduce the scope of the project to attribute more time to this task \\
    \end{tabular}
    \caption{Risks details}
    \label{tab:risk-assesment}
\end{table}
% TODO more risks